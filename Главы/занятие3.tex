\section{Источники света. Введение в теорию полупроводников.}
Лампа --- \hrulefill

\hrulefill

\hrulefill
\\
Светодиод --- \hrulefill

\hrulefill

\hrulefill\\
%полупроводниковый прибор с электронно-дырочным (PN) переходом, создающий оптическое излучение при пропускании через него электрического тока в прямом направлении.




\subsection{PN-переход. Введение}

Диод --- \hrulefill

\hrulefill

\hrulefill
%двухэлектродный электронный компонент, обладающий различной электрической проводимостью в зависимости от полярности приложенного к диоду напряжения.
\\
Полупроводники --- \hrulefill

\hrulefill
\\
Полупроводник N-типа --- \hrulefill

\hrulefill
\\
Полупроводник P-типа --- \hrulefill

\hrulefill
\\
Прямое и обратное подключение диода:

\begin{figure}[h]
\begin{minipage}[h]{0.5\linewidth}
\center{\begin{circuitikz}
\draw[line width = 0.3mm] (-1,1) -- (0,1); \draw[line width = 0.3mm] (6,1) -- (7,1);
\draw[line width = 0.3mm] (3,2) -- (3,0);
\draw[line width = 0.3mm] (-1,1) -- (-1,-1) to[battery] (7,-1) -- (7,1); 
\draw[line width = 0.3mm] (0,0) rectangle (6,2);
\end{circuitikz}}
\end{minipage}
\hfill
\begin{minipage}[h]{0.5\linewidth}
\center{\begin{circuitikz}
\draw[line width = 0.3mm] (-1,1) -- (0,1); \draw[line width = 0.3mm] (6,1) -- (7,1);
\draw[line width = 0.3mm] (3,2) -- (3,0);
\draw[line width = 0.3mm] (-1,1) -- (-1,-1) to[battery, invert] (7,-1) -- (7,1); 
\draw[line width = 0.3mm] (0,0) rectangle (6,2);
\end{circuitikz}}
\end{minipage}
\end{figure}

Главное свойство диода/светодиода --- \hrulefill

\hrulefill

\subsection{Параллельное и последовательное подключение светодиода.}

\begin{enumerate}
    \item Соберите по очереди схемы, указанные на рисунке \ref{ris:3.1}.
    \itemИзмерьте ЭДС батареи.
    \itemУкажите направление тока при всех замкнутых кнопках.
    \itemЗамкните ключ К.
    \itemДля каждого случая выполните следующее задание:
    \begin{enumerate}
        \itemПоочередно замыкая ключи (только для первого случая), измерьте напряжение и силу тока на каждом светодиоде.
    \itemПоменяйте полярность источника на обратную.
    \itemВыполните пункт (а) еще раз.
    \end{enumerate}
    \itemЗанесите все результаты в таблицу \ref{tab:3.1}.
    \itemСделайте выводы по наблюдениям.
    
\end{enumerate}


\begin{figure}[h]
\begin{minipage}[h]{0.5\linewidth}
\center{\begin{circuitikz}[european] \draw
(0,0) to[battery,l=$\mathscr{E}$, invert] (0,4) to[R, l=$100~\text{Ом}$] (2,4) --(6,4)to[empty led, l=$\text{Кр}$,fill=red](6,2) to[push button, l=$K_3$](6,0) --(1,0) to[normal open switch, l=$K$, mirror](0,0) 
(4,4)to[empty led, l=$\text{Жел}$,fill=yellow](4,2) to[push button, l=$K_2$](4,0)
(2,4)to[empty led, l=$\text{Зел}$,fill=green](2,2) to[push button, l=$K_1$](2,0)
;
\end{circuitikz}\\а.}
\end{minipage}
\hfill
\begin{minipage}[h]{0.5\linewidth}
\center{\begin{circuitikz}[european] \draw
(0,0) -- (0,0.4) to[battery, invert] (0,1)to[battery, l=$\mathscr{E}$, invert] (0,1.6) -- (0,2) to[R, l=$100~\text{Ом}$] (2,2) to[empty led, l=$\text{Кр}$,fill=red] (3.3,2) to[empty led, l=$\text{Жел}$,fill=yellow] (4.6,2)to[empty led, l=$\text{Зел}$,fill=green] (6,2) --
  (6,2)-- (6,0) to[normal open switch, l=$K$, mirror](0,0)
;
\end{circuitikz}\\ б.}
\end{minipage}
\caption{Параллельное и последовательное подключение светодиода}
\label{ris:3.1}
\end{figure}

\begin{table}[h]
 \caption{Прямое и обратное подключение светодиода.}
    \centering
    \begin{tabular}{|c|c|c|c|c|c|}
        \hline
        Подключение  & Полярность & ЭДС батареи $\mathscr{E}$, В & $U_\text{кр}$, В & $U_\text{жел}$, В & $U_\text{зел}$, В   \\
     
        \hline
       & Прямая  &   &  &  &  \\
        \cline{2-6}
        
        \raisebox{1.5ex}[0cm][0cm]{Параллельно}
        &Обратная  &   &  &  &  \\
        \hline

         & Прямая  &   &  &  &  \\
        \cline{2-6}
        \raisebox{1.5ex}[0cm][0cm]{Последовательно}
        &Обратная  &   &  &  &  \\
        \hline
    \end{tabular}
    \label{tab:3.1}
\end{table}

Вывод --- \hrulefill

\hrulefill

\hrulefill

\subsection{Подключение светодиода с различным сопротивлением.}

\begin{enumerate}
    \itemСоберите схему, указанную на рисунке \ref{fig:3.2}.
    \itemПри замкнутом ключе измерьте напряжение на каждом светодиоде.
    \itemЗанесите результаты в таблицу \ref{tab:3.2}.
    \itemЗапишите вывод о зависимости яркости светодиода от номинала сопротивления.
\end{enumerate}

\begin{figure}[h]
    \centering
    \begin{circuitikz}[european] \draw
(0,0)  to[battery, l= $\mathscr{E}$, invert](0,4)
  -- (8,4)  to[empty led, l=$\text{Кр}_4$, fill=red] (8,2) to[R, l=$100~\text{кОм}$] (8,0)--(2,0)
  to[normal open switch=$K$, mirror](0,0)
  (6,4)  to[empty led, l=$\text{Кр}_3$,fill=red] (6,2) to[R, l=$10~\text{кОм}$] (6,0)
  (4,4)  to[empty led, l=$\text{Кр}_2$,fill=red] (4,2) to[R, l=$1~\text{кОм}$] (4,0)
  (2,4)  to[empty led, l=$\text{Кр}_1$,fill=red] (2,2) to[R, l=$100~\text{Ом}$] (2,0)
;
\end{circuitikz}
    \caption{Подключение светодиода с различным сопротивлением}
    \label{fig:3.2}
\end{figure}

\begin{table}[h]
    \centering
    \caption{Зависимость яркости светодиода от сопротивления.}
    \begin{tabular}{|c|c|c|c|c|}
    
    \hline
    Схема & $U_1$, В &$U_2$, В&$U_3$, В&$U_4$, В \\
    \hline
     & & & & \\
    \raisebox{1.5ex}[0cm][0cm]{Ключ замкнут}
    & & & & \\
    \hline
    \end{tabular}
    
    \label{tab:3.2}
\end{table}

Вывод --- \hrulefill

\hrulefill

\hrulefill

\subsection{Поочередное свечение лампы и светодиода.}

\begin{enumerate}
    \item Соберите схему, указанную на рисунке \ref{fig:3.3}.
    \itemЗамкните ключ К1, пронабдюдайте горение лампы и светодиода. 
    \itemИзмерьте силу тока и напряжение на лампе и светодиоде. Занесите результаты в таблицу.
    \itemЗамкните ключ К1 и К2, пронаблюдайте горение лампы и светодиода.
    \itemИзмерьте силу тока и напряжение на лампе и светодиоде. Занесите результаты в таблицу \ref{tab:3.3}.
    \itemЗапишите вывод о том, почему в одном случае горит только лампа, а в другом только светодиод.
\end{enumerate}

\begin{figure}[h]
    \centering
    \begin{circuitikz}[european] \draw
(0,0) to[lamp, l=$\text{Л}$] (0,2) to[battery,l=$\mathscr{E}$, invert] (0,4) to[normal open switch, l=$K_1$](3,4)--(6,4) to[push button, l=$K_2$] (6,0)--(0,0)
(3,4)to[empty led, l=$\text{Кр}$, fill=red] (3,2) to[R, l=$100~\text{Ом}$](3,0)

;
\end{circuitikz}
    \caption{Поочередное свечение светодиода и лампы}
    \label{fig:3.3}
\end{figure}

\begin{table}[h]
    \centering
    \caption{Поочередное свечение лампы и светодиода.}
    \begin{tabular}{|c|c|c|c|c|}
    \hline
        Схема & Сила тока & Сила тока &Напряжение  &Напряжение \\
        & в лампе $I_\text{л}$, А & в светодиоде $I_\text{св}$, А& на лампе $U_\text{л}$, В&на светодиоде $U_\text{св}$, В \\
         \hline
         К1 замкнут& & & & \\
         К2 разомкнут& & & & \\
         \hline
         К1 замкнут& & & & \\
         К2 замкнут& & & & \\
         \hline
    \end{tabular}
    
    \label{tab:3.3}
\end{table}

Вывод --- \hrulefill

\hrulefill

\hrulefill

\newpage


