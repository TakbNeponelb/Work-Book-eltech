\newpage

\section{Резисторы, реостаты и потенциометры.}


\subsection{Работа и мощность}

Закон Джоуля-Ленца --- \hrulefill

\hrulefill

\hrulefill

Закона Джоуля-Ленца:

\begin{tikzpicture}
\draw (0,0) rectangle (6,2);
\end{tikzpicture}
\\
Мощность --- \hrulefill

\hrulefill

\hrulefill

Формула Мощности:

\begin{tikzpicture}
\draw (0,0) rectangle (6,2);
\end{tikzpicture}

\subsection{Резистор}

Резистор --- \hrulefill

\hrulefill

\hrulefill

Формула сопротивления проводника:

\begin{tikzpicture}
\draw (0,0) rectangle (6,2);
%$R = \rho fraq{l}{S}$%
\end{tikzpicture}
 
\subsection{Реостат и потенциометр}

Реостат --- \hrulefill%переменный резистр, управляющий током, изменяя сопротивление в цепи.

\hrulefill
\\
Потенциометр --- \hrulefill %переменный резистор, действующий как переменный делитель напряжения.

\hrulefill
\\
Различия и применение:
\begin{enumerate}
\item  \hrulefill
%Реостаты обычно имеют две клеммы, тогда как потенциометры имеют три для более точного управления.

\hrulefill
\item \hrulefill 
%Потенциометры находят применение в различных приложениях, таких как регулировка громкости и определение положения. Напротив, реостаты регулируют мощность в сильноточных приложениях, таких как диммеры и регуляторы скорости двигателя.

\hrulefill
\end{enumerate}

\newpage

\subsection{Резистор как ограничитель тока}
\begin{enumerate}
    \itemСоберите схему, указанную на рисунке \ref{fig:4.1}. Укажите в ней направление тока при замкнутом выключателе.
    \itemИзмерьте значение ЭДС батареи и занесите это значение в таблицу.
    \itemДля каждого сопротивления $R_1, R_2, R_3$ выполните следующее задание:
    \begin{enumerate}
        \itemИзмерьте силу тока в цепи $I_\text{изм}$.
        \itemРассчитайте силу тока с помощью закона Ома $I_\text{расч}$.
	\item Расчитайте, какая мощность рассеивается на резисторе с помощью формулы $P = UI$.
        \itemЗанесите данные в таблицу \ref{tab:4.1}.
    \end{enumerate}
    \itemЗапишите выводы о зависимости силы тока в цепи от ее внешнего сопротивления. 
\end{enumerate}

\begin{figure}[h]
    \centering
    \begin{circuitikz}[european] \draw
(0,0)to[battery, l=$\mathscr{E}$, invert] (0,3) to[R, l=$R$] (4,3) -- (4,0) to[normal open switch, l=$K$, mirror](0,0)
;
\end{circuitikz}
    \caption{Резистор как ограничитель тока}
    \label{fig:4.1}
\end{figure}

\begin{table}[h]
\centering
\caption{Резистор как ограничитель тока}
\begin{tabular}{|c|c|c|c|c|}
\hline
Схема & 
\begin{tabular}{@{}c@{}}
ЭДС батареи\\
$\mathscr{E}$, В 
\end{tabular} & 
\begin{tabular}{@{}c@{}}
Расчетная сила тока \\ в резисторе
$I_\text{расч}, мА$
\end{tabular} & 
\begin{tabular}{@{}c@{}}
Измеренная сила тока \\в резисторе
$I_\text{изм}, мА$
\end{tabular} & 
\begin{tabular}{@{}c@{}}
Мощность\\
на резисторе $P$, Вт
\end{tabular} \\
\hline
$R_1 = 1~kOm$   & & & & \\
\hline
$R_2 = 5.1~kOm$ & & & & \\
\hline
$R_3 = 56~kOm$  & & & & \\
\hline
\end{tabular}
\label{tab:4.1}
\end{table}

Вывод --- \hrulefill

\hrulefill

\hrulefill

\subsection{Реостат как ограничитель тока}
\begin{enumerate}
    \itemСоберите схему, указанную на рисунке \ref{fig:4.2}. Укажите в ней направление тока при замкнутом ключе.
    \itemВыполните следующее задание для реостатов с пределом $1$ кОм и $10$ кОм:
    \begin{enumerate}
        \itemЗамкните ключ К. 
        \itemУстановите ползунок реостата в крайнее левое положение А. 
        \itemДвигая ползунок реостата из крайнего левого положения А в крайнее правое положение В. Измерьте сопротивление реостата между точками А и С с помощью омметра, а также силу тока в цепи для пяти различных положение движка реостата (включая крайнее левое и крайнее правое). 
        \itemРезультаты измерений занесите в таблицу \ref{tab:4.2}
    \end{enumerate}
    \itemЗапишите вывод о том, как изменяется сила тока в цепи при движении ползунка реостата из крайнего левого положения А в крайнее правое положение В. 
\end{enumerate}

\begin{figure}[h]
    \centering
    \begin{circuitikz}[european] \draw
(0,0) -- (0,2) to[pR, name=P] (3,2);
\draw(0,0) to[normal open switch, l=$K$](6,0) to[battery, l=$\mathscr{E}$, invert] (6,4) to[R, l = $R$, a=$100~\text{Ом}$](2,4) --(1.5,4)-- (P.wiper);
\node[above,xshift=-2mm,yshift=-2mm] at (P.wiper) {B};
\node[left,xshift=-5mm,yshift=-8mm] at (P.wiper) {A};
\node[right, xshift=5mm,yshift=-8mm] at (P.wiper) {C};
%\draw (P1.wiper) \node[right]  {P1};
;
\end{circuitikz}
    \caption{Реостат как ограничитель тока}
    \label{fig:4.2}
\end{figure}

\begin{table}[h]
\caption{Реостат как ограничитель тока}
\centering
\begin{tabular}{@{}|c|c|c|c|c|c|@{}}
\hline
№ & \raisebox{1ex}[0cm][0cm]{Положение движка реостата}& \begin{tabular}[c]{@{}l@{}}Сопротивление \\ реостата, R Ом\end{tabular} & \begin{tabular}[c]{@{}l@{}}Сила тока\\ в цепи, I А\end{tabular} & \begin{tabular}[c]{@{}l@{}}Сопротивление \\ реостата, R Ом\end{tabular} & \begin{tabular}[c]{@{}l@{}}Сила тока\\ в цепи, I А\end{tabular} \\
\cline{3-6}
  & & \multicolumn{2}{|c|}{1 кОм} & \multicolumn{2}{c|}{10 кОм}  \\
  \hline
1 & крайнее левое & & & & \\
\hline
2 & промежуточное & & & & \\
\hline
3 & промежуточное&  & &  & \\
\hline
4 & промежуточное &  &  & & \\
\hline
5 & крайнее правое & & &  &\\
\hline
\end{tabular}
\label{tab:4.2}
\end{table}

Вывод --- \hrulefill

\hrulefill

\hrulefill

\subsection{Реостат как делитель напряжения}
\begin{enumerate}
    \itemСоберите схему, указанную на рисунке \ref{fig:4.3}.
    \itemЗамкните переключатель К. Поставьте движок реостата в крайнее верхнее положение А. Укажите направление силы тока в ней.
    \itemДвигая движок реостата из крайнего верхнего положения в крайнее нижнее, измерьте напряжения на светодиодах для четырех различных положений движка реостата (включая крайнее верхнее и крайнее нижнее). 
    \itemРезультаты измерений занесите в таблицу \ref{tab:4.3}. 
    \itemКакие светодиоды горят и от чего зависит яркость их свечения? Запишите ответ в выводе.
    \itemЗапишите вывод о возможности использования реостата в качестве делителя напряжения.
\end{enumerate}



\begin{figure}
    \centering
    \begin{circuitikz}[european]
    \draw (0,0) to[battery, invert,l=$\mathscr{E}$] (0,4) to[R, l = $R$, a=$100~\text{Ом}$](4,4) to[pR, name=P](4,0) to[normal open switch, mirror, l=$K$](0,0);
    \draw (4,4) -- (6,4) to[empty led, fill=red] (6,2) to[empty led, fill=green](6,0) -- (4,0);
    \draw (6,2) -- (P.wiper);
    \node[right, xshift=-5mm,yshift=-8mm] at (P.wiper) {C};
    \node[right, xshift=-1mm,yshift=2mm] at (P.wiper) {B};
    \node[right, xshift=-5mm,yshift=8mm] at (P.wiper) {A};
\end{circuitikz}
    \caption{Делитель напряжения}
    \label{fig:4.3}
\end{figure}

\newpage

\begin{table}[h]
\centering
\caption{Реостат как делитель напряжения}
\begin{tabular}{|c|c|cc|cc|}
\hline
\multirow{2}{*}{№} & \multirow{2}{*}{\begin{tabular}[c]{@{}c@{}}Положение\\ движка реостата\end{tabular}} & \multicolumn{2}{c|}{Красный светодиод} & \multicolumn{2}{c|}{Зеленый светодиод} \\ \cline{3-6} 
 & & \multicolumn{1}{c|}{\begin{tabular}[c]{@{}c@{}}Напряжение \\ $U_\text{кр}$, В\end{tabular}} & \begin{tabular}[c]{@{}c@{}}Горит \\ да/нет\end{tabular} & \multicolumn{1}{c|}{\begin{tabular}[c]{@{}c@{}}Напряжение\\ $U_\text{зел}$, В\end{tabular}} & \begin{tabular}[c]{@{}c@{}}Горит\\ да/нет\end{tabular} \\ 
 \hline
1& Крайнее левое& \multicolumn{1}{c|}{} && \multicolumn{1}{c|}{}&\\ 
\hline
2 & промежуточное& \multicolumn{1}{c|}{}& & \multicolumn{1}{c|}{}&                                                        \\ \hline
3 & промежуточное& \multicolumn{1}{c|}{} & & \multicolumn{1}{c|}{} &                                                        \\ \hline
4& промежуточное& \multicolumn{1}{c|}{}&  & \multicolumn{1}{c|}{}&                                                        \\ \hline
5& Крайнее правое & \multicolumn{1}{c|}{}  & & \multicolumn{1}{c|}{}&                                                        \\ \hline
\end{tabular}
\label{tab:4.3}
\end{table}

Вывод --- \hrulefill

\hrulefill

\hrulefill

\subsection{Двойной делитель напряжения}
\begin{enumerate}
    \itemСоберите схему, указанную на рисунке \ref{fig:4.4}.
    \itemЗамкните выключатель К. Регулируя положения движков реостатов, установите их в таком положении, чтобы горели все светодиоды (при необходимости добавьте 1-2 батарейных отсека). 
    \itemУкажите направление токов в схеме.
    \itemИзмерьте значения напряжений на светодиодах, а также ЭДС батареи. 
    \itemРезультаты измерения занесите в таблицу \ref{tab:4.4}. \itemПроверьте выполнение равенства $\mathscr{E}=U_\text{кр}+U_\text{жел}+U_\text{зел}$.
    \itemЗапишите вывод о возможности использования реостатов в качестве управляющих элементов по распределению напряжения в нагрузках.
\end{enumerate}

\newpage

\begin{figure}[h]
    \centering
    \begin{circuitikz}[european]
    \draw (0,0)--(0,2)to[battery, invert,l=$\mathscr{E}$] (0,3)to[battery, invert,l=$\mathscr{E}$](0,4)--(0,6) 
    
    to[R, l = $R$, a=$100~\text{Ом}$](4,6)--(4,5) to[pR, name=P1, a=$1~\text{кОм}$](4,3)to[pR, name=P2, a=$1~\text{кОм}$](4,1) --(4,0)to[normal open switch, mirror, l=$K$](0,0);
    
    \draw (4,6) -- (6,6) to[empty led, fill=red] (6,4) to[empty led, fill=yellow](6,2)to[empty led, fill=green] (6,0) --(4,0);
    \draw (6,2) -- (P2.wiper);
    \draw (6,4) -- (P1.wiper);
\end{circuitikz}
    \caption{Двойной делитель напряжения}
    \label{fig:4.4}
\end{figure}

\begin{table}[h]
\centering
\caption{Двойной делитель напряжения}
\begin{tabular}{|c|c|c|c|}
\hline
\begin{tabular}[c]{@{}c@{}}ЭДС батареи\\ $\mathscr{E}$, В\end{tabular}  & \begin{tabular}[c]{@{}c@{}}Напряжение\\ $U_\text{кр}$, В\end{tabular} & \begin{tabular}[c]{@{}c@{}}Напряжение\\ $U_\text{жел}$, В\end{tabular} & \begin{tabular}[c]{@{}c@{}}Напряжение\\ $U_\text{зел}$, В\end{tabular} \\ \hline
 & & & \\ \hline
\end{tabular}
\label{tab:4.4}
\end{table}

Вывод --- \hrulefill

\hrulefill

\hrulefill

