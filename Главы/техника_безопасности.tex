\section{Техника безопасности}


Учебная мастерская - это учебное помещение, где размещен
ручной инструмент, приспособления, станки и верстаки. С их
помощью ты моделируешь, конструируешь и изготовляешь
различные изделия или изучаешь свойства материалов.
Работая в учебных мастерских, помни о следующих
требованиях:

\subsection{Правила поведения в учебной мастерской:}

\begin{enumerate} 
\item На урок приходить вовремя.
\item  В учебную мастерскую входить только со звонком.
\item На урок приходить подготовленным: с собой иметь
письменные принадлежности, тетрадь, и тд.
\item После звонка занять своё рабочее место. Соблюдать
дисциплину во время урока.
\itemСвоё рабочее место содержать в чистоте и порядке. Не
загромождать проходы сумками и портфелями.
\itemСоблюдать правила пожарной безопасности.
\itemСоблюдать правила личной гигиены и санитарные нормы.
\itemСоблюдать правила поведения в учебной мастерской.
\itemСоблюдать правила электробезопасности: 
\begin{enumerate} 
\itemзапрещается
самовольно вкл/выкл центральный электрощит; 
\item запрещается использовать электроустановки без разрешения
учителя;
\item запрещается использовать бытовые
электроприборы без разрешения учителя.
\end{enumerate}
\itemПо территории мастерской передвигаться только шагом, не
менять рабочее место без разрешения учителя.
\end{enumerate}


\subsection{Правила безопасности в учебной мастерской:}

\begin{enumerate}
\itemБеспрекословно выполнять указания учителя.
\itemТребования к одежде:
\begin{enumerate}
\item одежда должна быть без внешних разрывов; 
\item должна быть в прилежном состоянии, иметь все
пуговицы и т.д.
\end{enumerate}
\item Запрещается носить колющие и режущие предметы в
карманах.

\item Запрещается передавать колющие и режущие предметы
режущей стороной вперёд.

\item Во избежание травм запрещается оставлять инструменты на
краю верстка.

\item Работать только исправным инструментом.

\item При нахождении неисправного инструмента немедленно
сообщить дежурному или учителю.

\item Бережно и аккуратно относиться к инструментам и
приспособлениям.
\item В случае ранения, а также при недомогании немедленно
обращаться за помощью к учителю.

\item Запрещается брать голыми руками горячие предметы,
оголенные провода, электрические розетки, кабельные
соединения.
\end{enumerate}



\textbf{Не знаешь — не лезь.}

Каждый, кто хочет приступить к работе с
электричеством,
обязан знать немного физики, математики и химии,
но много
техники безопасности, а также много-много всяких
мелочей.
Все это составляет общую картину мира
Электричества, который считается одной из малоизученных сфер науки.
Мы не
знаем доподлинно обо всех процессах и явлениях,
потому что
попросту не видим всего, что происходит в
электрооборудовании и его частях.
Многое принято как должное, или доказано по
косвенным
признакам и измерениям, ведь мы не может
пощупать электрическое поле и увидеть напряжение.

\textbf{Электричество — это ОЧЕНЬ опасно, но безумно интересно!}

\newpage