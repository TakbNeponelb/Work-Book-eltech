\newpage

\section{Соединение элементов. Нелинейные элементы.}


\subsection{Вольт-амперная характеристика (ВАХ)}

ВАХ --- \hrulefill

\hrulefill

\hrulefill\\
Линейные элементы --- \hrulefill

\hrulefill

\hrulefill\\
Нелинейные элементы --- \hrulefill

\hrulefill

\hrulefill



\subsection{Последовательное соединение резисторов}

\begin{enumerate}
    \itemСоберите схему, представленную на рисунке \ref{fig:5.1}.
    \itemВыполните следующее задание при замкнутом и разомкнутом ключе К2:
    \begin{enumerate}
        \itemЗамкните выключатель К1. Горит ли светодиод и почему? Измерьте силу тока в светодиоде. Результаты измерений занесите в таблицу \ref{tab:5.1}.
        \itemУкажите направление силы тока в схеме.
        \itemРассчитайте общее сопротивление цепи, используя измеренные значения ЭДС и силы тока в цепи. Результаты расчетов занесите в таблицу \ref{tab:5.1}. Проверьте выполнение равенства $R_\text{общ}=R_1+R_2$.
    \end{enumerate}
    \itemЗапишите выводы о характере зависимости общего сопротивления цепи от сопротивлений ее частей при их последовательном соединении. Запишите также объяснение выполнения/невыполнения равенства в пункте \ref{2c}.
\end{enumerate}

\begin{figure}[h]
    \centering
    \begin{circuitikz}[european]
        \draw (0,0) to[battery, invert] (0,2) to[empty led, fill = red](8,2) -- (8,0) to[R, l = $100~\text{кОм}$](6,0) to[R, l = $100~\text{Ом}$](3,0) to[normal open switch, l = $K_1$, mirror](0,0);
        \draw (8,1) to[push button, a = $K_2$,mirror] (6,1) -- (6,0);
    \end{circuitikz}
    \caption{Последовательное соединение резисторов}
    \label{fig:5.1}
\end{figure}


\begin{table}[h]
\centering
\caption{Последовательное соединение резисторов}
\label{tab:5.1}
\begin{tabular}{|c|c|c|c|c|c|}
\hline
Схема        & \begin{tabular}[c]{@{}c@{}}ЭДС батареи\\ $\mathscr{E}$, В\end{tabular} & \begin{tabular}[c]{@{}c@{}}Сила тока \\ в цепи\\ $I$, мА\end{tabular} & \begin{tabular}[c]{@{}c@{}}Напряжение\\ $U_\text{св}$, В\end{tabular} & \begin{tabular}[c]{@{}c@{}}Сопротивление \\светодиода $R_\text{св}$, Ом\end{tabular} & \begin{tabular}[c]{@{}c@{}}Горит светодиод?\\ да/нет\end{tabular} \\ \hline
$K_2$ разомкнут &                                                                        &                                                                       &                                                                        &                                                                                      &                                                                   \\ \hline
$K_2$ замкнут   &                                                                        &                                                                       &                                                                        &                                                                                      &                                                                   \\ \hline
\end{tabular}
\end{table}

Вывод --- \hrulefill

\hrulefill

\hrulefill



\subsection{Параллельное соединение резисторов}

\begin{enumerate}
    \itemСоберите схему, представленную на рисунке \ref{fig:5.2}.
    \itemВыполните следующее задание при замкнутом и разомкнутом ключе $K_2$:
    \begin{enumerate}
        \itemЗамкните выключатель $K_1$. Горит ли светодиод и почему? Измерьте силу тока в светодиоде. Результаты измерений занесите в таблицу \ref{tab:5.2}.
        \itemУкажите направление силы тока в схеме.
        \itemРассчитайте общее сопротивление цепи, используя измеренные значения ЭДС и силы тока в цепи. Результаты расчетов занесите в таблицу \ref{tab:5.2}. Проверьте выполнение равенства $\frac{1}{R_\text{общ}}=\frac{1}{R_1}+\frac{1}{R_2}$.\label{2c}
    \end{enumerate}
    \itemЗапишите выводы о характере зависимости общего сопротивления цепи от сопротивлений ее частей при их последовательном соединении. Запишите также объяснение выполнения/невыполнения равенства в пункте \ref{2c}.
\end{enumerate}

\begin{figure}[h]
    \centering
    \begin{circuitikz}[european]
        \draw (0,0) to[battery, invert] (0,2) to[empty led, fill = red](6,2) -- (6,0)  to[R, a = $100~\text{Ом}$](3,0) to[normal open switch, a = $K_1$, mirror](0,0);
        \draw (6,0)--(6,-1)to[R, a = $100~\text{кОм}$](3,-1) to[push button, a = $K_2$,mirror] (0,-1) --(0,0);
        %\draw (8,1)  (6,1) -- (6,0);
    \end{circuitikz}
    \caption{Параллельное соединение резисторов}
    \label{fig:5.2}
\end{figure}

\begin{table}[h]
\centering
\caption{Параллельное соединение резисторов}
\label{tab:5.2}
\begin{tabular}{|c|c|c|c|c|c|}
\hline
Схема        & \begin{tabular}[c]{@{}c@{}}ЭДС батареи\\ $\mathscr{E}$, В\end{tabular} & \begin{tabular}[c]{@{}c@{}}Сила тока \\ в цепи\\ $I$, мА\end{tabular} & \begin{tabular}[c]{@{}c@{}}Напряжение\\ $U_\text{св}$, В\end{tabular} & \begin{tabular}[c]{@{}c@{}}Сопротивление \\светодиода $R_\text{св}$, Ом\end{tabular} & \begin{tabular}[c]{@{}c@{}}Горит светодиод?\\ да/нет\end{tabular} \\ \hline
К2 разомкнут &                                                                        &                                                                       &                                                                        &                                                                                      &                                                                   \\ \hline
К2 замкнут   &                                                                        &                                                                       &                                                                        &                                                                                      &                                                                   \\ \hline
\end{tabular}
\end{table}

Вывод --- \hrulefill

\hrulefill

\hrulefill


\subsection{Лампа накаливания как нелинейный элемент}

\begin{enumerate}
    \itemСоберите схему, представленную на рисунке \ref{fig:5.3}. Укажите направление силы тока в ней при замкнутом ключе К.
    \itemСледуйте этому заданию до тех пор, пока в схеме не окажется 4 батарейных отсека.
    \begin{enumerate} 
    \itemИзмерьте силу тока в лампе и напряжение на ней при замкнутом ключе К. Результаты измерений занесите в таблицу \ref{tab:5.3}. 
    \itemПоменяйте полярность лампы и повторите пункт 2 (При этом значения напряжения и силы тока быть отрицательными).
    \itemДобавьте в схему еще один батарейный отсек.
    \end{enumerate}
    \itemИспользуя измеренные значения, рассчитайте сопротивление лампы в каждом случае и занесите его значение в таблицу \ref{tab:5.3}.
    \itemСравните полученные сопротивления лампы в различных опытах. Учитывая, что чем больше напряжение на лампе, тем больше яркость ее свечения, а значит и температура, сделайте предположение о том, как изменяется сопротивление лампы в зависимости от ее температуры? Запишите это в выводе.
    \item\textbf{Постройте в масштабе на миллиметровке вольт-амперную характеристику лампы накаливания} (график зависимости напряжения на лампе от силы протекающего через нее тока).
    \itemЗапишите вывод, в котором укажите: является ли лампа накаливания нелинейным элементом? 
\end{enumerate}

\begin{figure}[h]
    \centering
    \begin{circuitikz}
        \draw 
        (0,0) to[battery1, invert, l=$\mathscr{E}$] (0,3) -- (3,3) to[lamp, fill=yellow] (3,0) to[normal open switch, a=K, mirror] (0,0) 
        ;
    \end{circuitikz}
    \caption{Исследование лампы на нелинейность}
    \label{fig:5.3}
\end{figure}

% Please add the following required packages to your document preamble:
% \usepackage{multirow}
\begin{table}[h]
\centering
\caption{Исследование лампы на нелинейность}
\label{tab:5.3}
\begin{tabular}{|c|c|c|c|c|}
\hline
\begin{tabular}[c]{@{}c@{}}Количество\\ батарейных\\ отсеков\end{tabular} & Полярность & \begin{tabular}[c]{@{}c@{}}Сила тока\\ в лампе\\ I, мА\end{tabular} & \begin{tabular}[c]{@{}c@{}}Напряжение на\\ лампе\\ U, В\end{tabular} & \begin{tabular}[c]{@{}c@{}}Сопротивление\\ лампы\\ R, Ом\end{tabular} \\ \hline
\multirow{2}{*}{1}                                                        & прямая     &                                                                     &                                                                      &                                                                       \\ \cline{2-5} 
                                                                          & обратная   &                                                                     &                                                                      &                                                                       \\ \hline
\multirow{2}{*}{2}                                                        & прямая     &                                                                     &                                                                      &                                                                       \\ \cline{2-5} 
                                                                          & обратная   &                                                                     &                                                                      &                                                                       \\ \hline
\multirow{2}{*}{3}                                                        & прямая     &                                                                     &                                                                      &                                                                       \\ \cline{2-5} 
                                                                          & обратная   &                                                                     &                                                                      &                                                                       \\ \hline
\multirow{2}{*}{4}                                                        & прямая     &                                                                     &                                                                      &                                                                       \\ \cline{2-5} 
                                                                          & обратная   &                                                                     &                                                                      &                                                                       \\ \hline
\end{tabular}
\end{table}

Вывод --- \hrulefill

\hrulefill

\hrulefill


\subsection{Звонок как нелинейный элемент}

\begin{enumerate}
    \itemСоберите схему, представленную на рисунке \ref{fig:5.4}. Укажите направление силы тока в ней при замкнутом ключе К.
    \itemДвигая движок реостата из крайнего верхнего положения в крайнее нижнее, измерьте напряжение на звонке и силу тока в нем для пяти различных положений движка реостата для прямой и обратной полярности (при этом значения напряжения и силы тока быть отрицательными) включения звонка (включая крайнее верхнее и крайнее нижнее). Результаты измерений занесите в таблицу \ref{tab:5.4}. 
    \itemИспользуя измеренные значения, рассчитайте сопротивление звонка в каждом случае и занесите его значение в таблицу \ref{tab:5.4}.
    \itemСравните полученные сопротивления звонка в различных опытах. Как изменяется сопротивление звонка в зависимости от напряжения на нем и его полярности?
    \item \textbf{Постройте в масштабе на миллиметровке вольт-амперную характеристику звонка} (график зависимости напряжения на звонке от силы протекающего через нее тока).
    \itemЗапишите вывод, в котором укажите: является ли звонок нелинейным элементом? 
\end{enumerate}

\begin{figure}[h]
    \centering
    \begin{circuitikz}
        
        \draw (0,0) to[pR, name=R, a=$1~\text{кОм}$]  (2,0) to[buzzer, fill=black] ++(0,-1.5) to[normal open switch, a=K, mirror] ++(-4,0) to[battery, invert, l=$\mathscr{E}$] ++ (0,2.5) -| (R.wiper);
        
    \end{circuitikz}
    \caption{Исследование звонка на нелинейность}
    \label{fig:5.4}
\end{figure}

% Please add the following required packages to your document preamble:
% \usepackage{multirow}
\begin{table}[h]
\centering
\caption{Исследование звонка на нелинейность}
\label{tab:5.4}
\begin{tabular}{|c|c|c|c|c|}
\hline
\begin{tabular}[c]{@{}c@{}}Положение\\ движка\\ реостата\end{tabular} & Полярность & \begin{tabular}[c]{@{}c@{}}Сила тока\\ в лампе\\ I, мА\end{tabular} & \begin{tabular}[c]{@{}c@{}}Напряжение на\\ лампе\\ U, В\end{tabular} & \begin{tabular}[c]{@{}c@{}}Сопротивление\\ лампы\\ R, Ом\end{tabular} \\ \hline
\multirow{2}{*}{1}                                                    & прямая     &                                                                     &                                                                      &                                                                       \\ \cline{2-5} 
                                                                      & обратная   &                                                                     &                                                                      &                                                                       \\ \hline
\multirow{2}{*}{2}                                                    & прямая     &                                                                     &                                                                      &                                                                       \\ \cline{2-5} 
                                                                      & обратная   &                                                                     &                                                                      &                                                                       \\ \hline
\multirow{2}{*}{3}                                                    & прямая     &                                                                     &                                                                      &                                                                       \\ \cline{2-5} 
                                                                      & обратная   &                                                                     &                                                                      &                                                                       \\ \hline
\multirow{2}{*}{4}                                                    & прямая     &                                                                     &                                                                      &                                                                       \\ \cline{2-5} 
                                                                      & обратная   &                                                                     &                                                                      &                                                                       \\ \hline
\multirow{2}{*}{5}                                                    & прямая     & \multicolumn{1}{l|}{}                                               & \multicolumn{1}{l|}{}                                                & \multicolumn{1}{l|}{}                                                 \\ \cline{2-5} 
                                                                      & обратная   & \multicolumn{1}{l|}{}                                               & \multicolumn{1}{l|}{}                                                & \multicolumn{1}{l|}{}                                                 \\ \hline
\end{tabular}
\end{table}

Вывод --- \hrulefill

\hrulefill

\hrulefill



\subsection{Диод как нелинейный элемент}

\begin{enumerate}
\itemСоберите схему, представленную на рисунке \ref{fig:5.5}. Укажите направление силы тока в ней при замкнутом ключе К.
    \itemДвигая движок реостата из крайнего верхнего положения в крайнее нижнее, измерьте напряжение на диоде и силу тока в нем для пяти различных положений движка реостата для прямой и обратной полярности (при этом значения напряжения и силы тока быть отрицательными) включения звонка (включая крайнее верхнее и крайнее нижнее). Результаты измерений занесите в таблицу \ref{tab:5.5}. 
    \itemИспользуя измеренные значения, рассчитайте сопротивление диода в каждом случае и занесите его значение в таблицу \ref{tab:5.5}.
    \itemСравните полученные сопротивления диода в различных опытах. Как изменяется сопротивление звонка в зависимости от напряжения на нем и его полярности?
    \item\textbf{Постройте в масштабе на миллиметровке вольт-амперную характеристику диода} (график зависимости напряжения на диоде от силы протекающего через нее тока).
    \itemЗапишите вывод, в котором укажите: является ли диод нелинейным элементом? 
\end{enumerate}

\newpage

\begin{figure}[h]
    \centering
    \begin{circuitikz}
        
        \draw (0,0) to[pR, name=R, a=$1~\text{кОм}$]  (2,0) to[full diode] ++(0,-1.5) to[normal open switch, a=K, mirror] ++(-4,0) to[battery, invert, l=$\mathscr{E}$] ++(0,2.5) -- ++(1,0)to[R, l=$200~\text{Ом}$]++(1,0)-| (R.wiper);
        
    \end{circuitikz}
    \caption{Исследование диода на нелинейность}
    \label{fig:5.5}
\end{figure}

\begin{table}[h]
\centering
\caption{Исследование диода на нелинейность}
\label{tab:5.5}
\begin{tabular}{|c|c|c|c|c|}
\hline
\begin{tabular}[c]{@{}c@{}}Положение\\ движка\\ реостата\end{tabular} & Полярность & \begin{tabular}[c]{@{}c@{}}Сила тока\\ в лампе\\ I, мА\end{tabular} & \begin{tabular}[c]{@{}c@{}}Напряжение на\\ лампе\\ U, В\end{tabular} & \begin{tabular}[c]{@{}c@{}}Сопротивление\\ лампы\\ R, Ом\end{tabular} \\ \hline
\multirow{2}{*}{1}                                                    & прямая     &                                                                     &                                                                      &                                                                       \\ \cline{2-5} 
                                                                      & обратная   &                                                                     &                                                                      &                                                                       \\ \hline
\multirow{2}{*}{2}                                                    & прямая     &                                                                     &                                                                      &                                                                       \\ \cline{2-5} 
                                                                      & обратная   &                                                                     &                                                                      &                                                                       \\ \hline
\multirow{2}{*}{3}                                                    & прямая     &                                                                     &                                                                      &                                                                       \\ \cline{2-5} 
                                                                      & обратная   &                                                                     &                                                                      &                                                                       \\ \hline
\multirow{2}{*}{4}                                                    & прямая     &                                                                     &                                                                      &                                                                       \\ \cline{2-5} 
                                                                      & обратная   &                                                                     &                                                                      &                                                                       \\ \hline
\multirow{2}{*}{5}                                                    & прямая     & \multicolumn{1}{l|}{}                                               & \multicolumn{1}{l|}{}                                                & \multicolumn{1}{l|}{}                                                 \\ \cline{2-5} 
                                                                      & обратная   & \multicolumn{1}{l|}{}                                               & \multicolumn{1}{l|}{}                                                & \multicolumn{1}{l|}{}                                                 \\ \hline
\end{tabular}
\end{table}

Вывод --- \hrulefill

\hrulefill

\hrulefill

