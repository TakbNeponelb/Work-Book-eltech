\newpage

\section{Катушка индуктивности}

\subsection{Предисловие}

\subsubsection{Вектор магнитной индукции}

Вектор магнитной индукции --- \hrulefill
% Векторная физическая величина, характеризующая магнитное поле
% $B = M/(IS) = F/(Il)$

\hrulefill

\hrulefill

Единицы измерения магнитной индукции:

\begin{tikzpicture}
\draw (0,0) rectangle (6,2);
% $B = 1~\text{Тл}$
\end{tikzpicture}


\subsubsection{Поток вектора магнитной индукции}

Поток вектора --- \hrulefill
%Потоком вектора магнитной индукции Φ через плоскую поверхность 
%площадью ∆S называют величину, равную произведению модуля вектора магнитной индукции →
%B на площадь S и косинус угла a между вектором →
%B и нормалью к поверхности → n.

\hrulefill

\hrulefill


Формула потока вектора магнитной индукции:

\begin{tikzpicture}
\draw (0,0) rectangle (6,2);
% $Ф = B*S*cos(a)$
\end{tikzpicture}

Единицы измерения потока:

\begin{tikzpicture}
\draw (0,0) rectangle (6,2);
\end{tikzpicture}

\subsubsection{Связь между потоком и силой тока}

Модуль индукции $B$ магнитного поля, создаваемого током в любом замкнутом контуре, пропорционален силе тока. Так как магнитный поток $\Phi$ пропорционален $B$, то 
$\Phi \sim B \sim I$.
\begin{center}
$\Phi = LI$
\end{center}

Где $L$  — \hrulefill

\hrulefill
%коэффициент пропорциональности (индуктивность контура, или коэффициент самоиндукции) между силой тока в проводящем контре и созданным им магнитным потоком, пронизывающим этот контур.

\subsection{Катушка индуктивности и индуктивность}

Катушка индуктивности --- \hrulefill

\hrulefill

\hrulefill

Обозначение катушки индуктивности в схемах:

\begin{tikzpicture}
\draw (0,0) rectangle (6,2);
\end{tikzpicture}

\newpage 

Индуктивность --- \hrulefill

\hrulefill

\hrulefill

Единица измерения индуктивности: 

\begin{tikzpicture}
\draw (0,0) rectangle (6,2);
\end{tikzpicture}

\subsection{Электромагнитная индукция}

Электромагнитная индукция --- \hrulefill
%Явление возникновения электрического тока в замкнутом контуре при изменении магнитного поля через площадь, ограниченную контуром.

\hrulefill

\hrulefill


\subsubsection{Закон Фарадея}

ЭДС индукции --- \hrulefill

\hrulefill

\hrulefill

Закон Фарадея:

\begin{tikzpicture}
\draw (0,0) rectangle (6,2);
% $\mathscr(E)_i = - fraq{\delta \Phi}{\delta t}$
\end{tikzpicture}

%ЭДС индукции в замкнутом контуре равна по модулю скорости изменения магнитного потока через поверхность, ограниченную контуром.




\subsection{Получение электричества при помощи катушки индуктивности и постоянного магнита}

\begin{enumerate}
    \itemСоберите схему, представленную на рисунке \ref{fig:6.1}.
    \itemУстановите предел измерения вольтметра 200 мВ.
    \itemПеремещайте магнит вдоль катушки и наблюдайте изменение показаний вольтметра. Объясните данное явление.
    \itemЗапишите выводы о том, как зависят показания вольтметра от скорости перемещения магнита и направления его перемещения?

\end{enumerate}

\begin{figure}[h]
    \centering
    \begin{circuitikz}[european]
	\draw (0,0) to[cute inductor] ++(4,0) -- ++(0,-2) to[rmeter, t=$V$] ++(-4,0) -- ++(0,2); 
    \end{circuitikz}
    \caption{Получение электричества при помощи катушки и магнита}
    \label{fig:6.1}
\end{figure}

Вывод --- \hrulefill

\hrulefill

\hrulefill

\subsection{Электромагнит}

\begin{enumerate}
    \itemСоберите схему, представленную на рисунке \ref{fig:6.2}. Укажите направление силы тока в ней с замкнутым выключателем $К_1$.
    \itemВыполните следующее задание при положении компаса сначала с одной стороны катушки, после с противоположной:
    \begin{enumerate}
         \itemПоложите магнитную стрелку (иголку) рядом с катушкой индуктивности. Замкните выключатель $К_1$ и наблюдайте за изменением положения магнитной стрелки. Объясните данное явление.
        \itemПри замкнутом выключателе $К_1$ кратковременно зажмите кнопку $К_2$ и наблюдайте за изменением положения магнитной стрелки. Объясните данное явление.
    \end{enumerate}
    \itemЗапишите выводы о магнитных свойствах катушки индуктивности.

\begin{figure}[h]
    \centering
    \begin{circuitikz}[european]
	\draw (0,0) to[battery1, invert](0,1)to[battery1, invert](0,2)to[battery1, invert](0,3)to[battery1, invert](0,4) coordinate(p1);
	\draw (p1) to[cute inductor, l=$L$] ++(3,0) coordinate(p3); 
	\draw (p3) to[lamp, l=$\text{Л}$] ++(2,0) -- ++(0,-2) coordinate(p4);
	\draw (0,0) to[normal open switch, l=$K_1$] ++(3,0) -| (p4);
	\draw (p3) -- ++(0,-2) to[push button, l=$K_2$] (p4);
	\draw[red,fill=red] (0.5,4.3) -- (0.7,4.4) -- (0.7,4.2) -- (0.51,4.3);
	\draw[blue,fill=blue] (0.71,4.2) -- (0.71,4.4) -- (0.91,4.3) -- (0.70,4.2);
	\draw[red,fill=red] (2,4.3) -- (2.2,4.4) -- (2.2,4.2) -- (2.01,4.3);
	\draw[blue,fill=blue] (2.21,4.2) -- (2.21,4.4) -- (2.41,4.3) -- (2.2,4.2);
    \end{circuitikz}
    \caption{Получение электричества при помощи катушки и магнита}
    \label{fig:6.2}
\end{figure}

Вывод --- \hrulefill

\hrulefill

\hrulefill

\end{enumerate}

\subsection{Проверка явления 
самоиндукции}

\begin{enumerate}
    \itemСоберите схему, представленную на рисунке \ref{fig:6.3}.
    \itemЗажмите кнопку К. Как горят светодиоды и почему? Укажите направления силы тока в данном случае.
    \itemОтпустите кнопку К. Как горят светодиоды и почему? Укажите направления силы тока в данном случае.
    \itemЗапишите выводы о протекании тока при зажатой и отжатой кнопке К, а также о возможности накопления энергии катушкой индуктивности.
\end{enumerate}

\newpage

\begin{figure}[h]
    \centering
    \begin{circuitikz}[european]
	\draw (0,0) -- (0,4) to[empty led, fill=green] (2,4) to[R, l=$1~\text{кОм}$] (4,4) to[cute inductor, l = $L$] (6,4) -- (6,0) to[normal open switch, mirror, a=$K$] (2.9,0) to[battery1, invert] (2.22,0)to[battery1, invert] (1.5,0)to[battery1, invert] (0.8,0)to[battery1, invert] (0.1,0)--(0,0);
	\draw (0,2) to[empty led, invert, fill=red] (3,2) to[R, l=$1~\text{кОм}$] (6,2);
    \end{circuitikz}
    \caption{Проверка явления самоиндукции}
    \label{fig:6.3}
\end{figure}

Вывод --- \hrulefill

\hrulefill

\hrulefill

